%
% 6.006 problem set 1
%
\documentclass[12pt,twoside]{article}

\input{macros}

\usepackage{amsmath}
\usepackage{url}
\usepackage{mdwlist}
\usepackage{graphicx}
\usepackage{enumitem}
%\usepackage{clrscode3e}

\setlength{\oddsidemargin}{0pt}
\setlength{\evensidemargin}{0pt}
\setlength{\textwidth}{6.5in}
\setlength{\topmargin}{0in}
\setlength{\textheight}{8.5in}

% Fill these in!
\newcommand{\theproblemsetnum}{1B}
\newcommand{\releasedate}{September 15, 2011}
\newcommand{\partaduedate}{Tuesday, September 20}
\newcommand{\tabUnit}{3ex}
\newcommand{\tabT}{\hspace*{\tabUnit}}

\begin{document}

\handout{Problem Set \theproblemsetnum}{\releasedate}

\textbf{Your critique is due  {\bf \partaduedate} at {\bf 11:59PM}.}

\setlength{\parindent}{0pt}

\medskip

\hrulefill

\begin{problems}

\problem \textbf{Proof Critique}

\paragraph{Previous Proof.}

%%% PREVIOUS PROOF START %%%
Previously on the 6.006, I only compare the asymptotic notation raguely.
%%% PREVIOUS PROOF END %%%

\paragraph{Critique.}

%%% CRITIQUE START %%%
Use different fucntion to transmite the inequel relation;
\subsection{1-1} a.I need to compare $f_1$ with $f_2$ and $f_3$ with $f_4$. I only look at the oder of growth. Truly, $nlog(n) = O(n)$. But for function 1 and 2, since $lg^b (n) = O(n^a)$ for any postive a, $n^{0.9999999}log(n) = O(n^{0.9999999} \cdot n^{0.0000000001}) = O(n) = O(f_2)$; \par
c. the only trouble is $f_1$. $f_1 = 2^{\sqrt{n}log(n)}$.However, $f_2 = 2^n$, $f_3 = x^{\frac{n}{2} + 10 log(n)}$. So $f_3 = O(f_2)$, $f_1 = O(f_3) .$ The Big-O notation of $f(n) = O(g(n))$ means some constant multiple of g(n) is the upper bound for f(n). 

\subsection{1-2}
This problem ask to compute the recursive computational complexity.  \par
a. expand the recursive tree to an geometric series, which is bounded by 2(x + y). Thus O(n) si the correct anwser.\par
b. almost the same procedure for 1-2(a); \par
c. solve the mutually recusive relation. $T(x, y) = O(x) + O(y/2) + T(x/2, y/2) = O(x + y) + T(x/2, y/2)$. Then the problem is reduced to problem 1-2 a;

\subsection{1-3, Peak Finding}
To deal with recursive algorithms, I need to look at problems more holistically, in other words, more complete and systematic instead of reduclistically.

%%% CRITIQUE END %%% 

\end{problems}

\end{document}
